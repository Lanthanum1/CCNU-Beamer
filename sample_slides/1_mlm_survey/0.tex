\section{MATHEMATICAL TASKS}
\subsection{Mathematical Calculation}
\begin{frame}{Arithmetic Representation}
	\begin{itemize}
		\item \textbf{Numerical Challenges}
		      \begin{itemize}
			      \item 初期处理数值时,常被忽略或简单化。
			      \item BERT在遇到数值答案时表现较差。
		      \end{itemize}
		      \pause
		\item \textbf{近期表示方法}
		      \begin{itemize}
			      \item GenBERT:
			            \begin{itemize}
				            \item 数字按位数进行标记。
				            \item 进行算术问题的微调。
			            \end{itemize}
			      \item 数字转换为科学计数法。
			      \item 使用数字嵌入形成整体的数字表示。
			      \item 使用Digit-RNN和指数嵌入,重点突出指数。
			      \item 引入一致的标记化方法,增强相似数值之间的关系。
		      \end{itemize}
	\end{itemize}
\end{frame}

\begin{frame}{Arithmetic Calculation}
	\begin{itemize}
		% \item 研究对象:加法、减法以及两位数乘法等算术任务。
		%       \pause
		% \item 传统上认为LLM难以进行复杂的算术运算,尤其是大位数乘法。
		%       \pause
		\item \textbf{近期方法}
		      \begin{itemize}
			      \item 应用 specialized prompt engineering 提升加法能力,但乘法有位数限制。
			            \pause
			      \item 利用 relative position embeddings 和 training set priming 实现算术任务的长度泛化。
			            \pause
			            \begin{itemize}
				            \item (回忆一下 Transformer 的 encode 过程)一般来说,Transformer模型使用绝对位置嵌入来为输入序列中的每个位置提供一个固定的位置标识符,这种方式在处理序列长度变化时表现不佳,因为这种嵌入将标记的位置信息与其自身的表示混合在一起,导致模型对序列长度的变化非常敏感。
				                  \pause
				            \item 相对位置嵌入则是用来编码标记之间的相对距离,而不是它们的绝对位置。这意味着模型可以更容易地将学习到的模式应用到未见过的序列长度上,因为相对位置信息可以更好地适应变化的序列长度。
				                  \pause
				            \item 对于乘法任务,相对位置嵌入并不能使模型实现长度上的泛化(仅仅依靠相对位置信息可能不足以捕捉到乘法运算的所有细节)。为了克服这一限制,作者提出了“训练集引导”的方法,通过在训练集中添加少量长序列的例子来帮助模型泛化到更长的序列长度上。例如,在训练集中包含极少的长序列样本(如50个样本),就能使原本只能处理短序列的模型扩展到处理35位数乘以3位数这样的长序列计算任务。
			            \end{itemize}

		      \end{itemize}
	\end{itemize}

\end{frame}

\begin{frame}{Arithmetic Calculation}
	\begin{itemize}
		\item ScratchpadGPT通过 CoT 在8位加法中表现出色。
		      \pause
		\item 监督学习用于微调大整数的基础运算。(将各种算术任务分为可学习和不可学习的任务,然后通过利用基本算术原则将不可学习的任务分解为一系列可学习的任务。)
		      \pause
		\item MathGLM通过在数据集中分解复杂算术表达式,逐步生成答案并学习计算规则。
	\end{itemize}
\end{frame}

\subsection{Mathematical Reasoning}
\begin{frame}
	\frametitle{Math Problem Solving}
	\begin{itemize}
		\item \textbf{Math Word Problem Solving}
		      \begin{itemize}
			      \item 数学文字问题 (MWPs) \textbf{通过文字描述来呈现数学概念和计算},要求从中提取相关数学信息并应用适当的数学原理。
			            %   \pause
			      \item 数学文字问题解决的研究重点在于\textbf{使用高效智能的算法解决问题}。
			            \pause
			      \item 近期研究:
			            \begin{itemize}
				            \item MathPrompter 使用 GPT-3 DaVinci 模型取得优异成绩,展现LLMs在复杂数学推理方面的潜力。
				                  % \pause
				            \item 拒绝采样微调 (RFT)提升推理泛化能力。(通过生成多个候选样本,选择其中高质量样本进行模型微调的训练方法)
				                  % \pause
				            \item MetaMath 通过从多个角度重新编写数学问题来生成新的数据集MetaMathQA,然后在此数据集上微调LLaMA-2模型
			            \end{itemize}
		      \end{itemize}
	\end{itemize}
\end{frame}
\begin{frame}
	\frametitle{Math Problem Solving}
	\begin{itemize}
		\item MathAttack:
		      \begin{itemize}
			      \item 逻辑实体识别: 找出题目中的关键元素,比如人物(Role Entity)、数值(Number Entity)以及场景(Scenario Entity,如时间或地点)。这些元素对于构成题目逻辑至关重要,因此一旦改变就可能会改变题目的实际意义。
			            %   \pause
			      \item 冻结逻辑实体: 这些重要的信息将不会被更改
			            %   \pause
			      \item 词级攻击: 不改变上述逻辑实体的前提下,对题目中的其他词汇进行替换或修改,以期影响模型对题目的理解和解答能力。
			            %   \pause
			      \item 这种方法旨在创造一个对抗样例,这个样例在人类看来与原题非常相似,但对于机器学习模型来说,它可能导致错误的理解或解答。提高了 LLMs 对于数学问题的鲁棒性。
		      \end{itemize}
		      \pause
		\item LLEMMA:
		      \begin{itemize}
			      \item \textbf{MATH+Python}: 模型需交替地用自然语言描述解题步骤,然后编写相应的Python代码来执行该步骤。最后的答案是一个数字结果或者SymPy对象。
		      \end{itemize}
	\end{itemize}
\end{frame}

% \begin{frame}
% 	\frametitle{Math Question Answering}
% 	\begin{itemize}
% 		\item 数学问题问答(MQA)是要求 LLMs 理解并自动解决自然语言形式的数学问题。涵盖从简单算术到复杂的高中或大学数学,包括代数、微积分和几何。
% 		\item \textbf{研究进展}
% 		      \begin{itemize}
% 			      \item Sachan等提出评估神经网络架构的框架,强调代数泛化的重要性。
% 			      \item GEOS系统:结合文本理解和图表解析来解决SAT几何问题。
% 			      \item 新数据集与系统:
% 			            \begin{itemize}
% 				            \item Inter-GPS、IconQA和PGDP5K等提供抽象图表理解和视觉语言推理的新基准。
% 				            \item Scharpf等研究无监督公式标记方法,促进数学文档的自动理解。
% 			            \end{itemize}
% 		      \end{itemize}
% 	\end{itemize}
% \end{frame}

\begin{frame}
\frametitle{Theorem Proving}
Baldur:
\begin{enumerate}
	\item \textbf{证明生成}: 将定理陈述作为输入,LLM 尝试生成完整的证明,然后使用 Isabelle 证明助手进行验证。
	\item \textbf{证明修复}: 如果验证结果表明证明失败,则使用 LLM 尝试新的证明,并再次使用 Isabelle 证明助手进行验证。
	\item \textbf{上下文信息}: 将定理所在的上下文信息作为补充输入,以帮助 LLM 生成更准确的证明。
\end{enumerate}
\pause
Draft, Sketch, and Prove(DSP):
\begin{enumerate}
\item \textbf{起草非正式证明}:从一个数学问题的非正式陈述开始,这个陈述通常是以自然语言和数学公式混合的形式描述的。
\item \textbf{映射成形式证明草图}:使用大型语言模型将每个非正式证明自动转换为形式证明草图。
\item \textbf{证明剩余的猜想}:使用这些形式证明草图来引导自动化证明器,导向更简单的子问题。
\end{enumerate}


\end{frame}