\section{MATHEMATICAL TASKS}
\subsection{Mathematical Calculation}
\begin{frame}{Arithmetic Representation}
	\begin{itemize}
		\item \textbf{Numerical Challenges}
		      \begin{itemize}
			      \item 初期处理数值时,常被忽略或简单化。
			      \item BERT在遇到数值答案时表现较差。
		      \end{itemize}
		\item \textbf{近期表示方法}
		      \begin{itemize}
			      \item GenBERT:
			            \begin{itemize}
				            \item 数字按位数进行标记。
				            \item 进行算术问题的微调。
			            \end{itemize}
			      \item 数字转换为科学计数法。
			      \item 使用数字嵌入形成整体的数字表示。
			      \item 使用Digit-RNN和指数嵌入,重点突出指数。
			      \item 引入一致的标记化方法,增强相似数值之间的关系。
		      \end{itemize}
	\end{itemize}
\end{frame}

\begin{frame}{Arithmetic Calculation}
	\begin{itemize}
		\item 研究对象:加法、减法以及两位数乘法等算术任务。

		\item \textbf{传统假设与挑战}
		      \begin{itemize}
			      \item 传统上认为LLM难以进行复杂的算术运算,尤其是大位数乘法。
			      \item 新方法的应用挑战了这一假设。
		      \end{itemize}
		\item \textbf{近期方法}
		      \begin{itemize}
			      \item 应用专项提示工程提升加法能力,但乘法有位数限制。
			      \item 利用相对位置嵌入和训练集优化研究算术任务的长度泛化。
			      \item ScratchpadGPT通过预生成思维链在8位加法中表现出色。
			      \item 监督学习用于微调大整数的基础运算。
			      \item MathGLM通过在数据集中分解复杂算术表达式,逐步生成答案并学习计算规则。
		      \end{itemize}
	\end{itemize}

\end{frame}
